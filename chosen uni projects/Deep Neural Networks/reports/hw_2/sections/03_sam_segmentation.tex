The Segment Anything Model (SAM) requires prompts (points, boxes) to segment objects. We automated this prompting process using the localization priors from our Grad-CAM implementation. We compared two approaches:

\subsection{Pipeline 1: Foreground Points Only}
(\texttt{SamForegroundPipeline}) relies on positive cues from Grad-CAM heatmaps:
\begin{itemize}
    \item \textbf{Point Selection:} We identify the peak activation in the heatmap and calculate the center of mass for regions exceeding 80\% of this maximum value.
    \item \textbf{Augmentation:} We generate a $3 \times 3$ grid of foreground points centered on this location (the center point plus its 8 neighbors with an offset of 1 pixel).
\end{itemize}

\subsection{Pipeline 2: Foreground and Background Points}
(\texttt{SamBackgroundPipeline}) adds negative constraints to refine the segmentation boundary.
\begin{itemize}
    \item \textbf{Foreground:} Identical to Pipeline 1.
    \item \textbf{Background:} We identify regions with very low activation (heatmap values $< 0.1$). From these "cold" regions, we select 4 spatially distinct points (top-most, bottom-most, left-most, right-most) to explicitly signal background areas to SAM.
\end{itemize}